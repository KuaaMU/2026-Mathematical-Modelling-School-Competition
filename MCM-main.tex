\documentclass{mcmthesis}
\mcmsetup{CTeX = false,    % 使用 CTeX 套装时,设置为 true
          tcn = {0000000},  % 参赛队编号
          problem = \textcolor{red}{A}, % 参赛问题编号 A, B, C, D
          sheet = true, titleinsheet = true, keywordsinsheet = true,
          titlepage = false, abstract = false}

\usepackage{newtxtext}     % \usepackage{palatino} % 使用 Palatino 字体
\usepackage{newtxmath}     % \usepackage{mathpazo} % 使用 Pazo Math 字体
\usepackage[style=apa,backend=biber]{biblatex} % 引用管理
\addbibresource{texfile/reference.bib}

\usepackage{tocloft}    % 目录设置
\setlength{\cftbeforesecskip}{6pt} % 目录项之间的间距
\renewcommand{\contentsname}{\hspace*{\fill}\Large\bfseries Contents \hspace*{\fill}}   % 目录标题居中加粗


\title{Enjoy a Cozy and Green Bath} %论文标题

% \author{\small \href{http://www.latexstudio.net/}
%   {\includegraphics[width=7cm]{mcmthesis-logo}}}
\date{\today} % 日期,默认今天


%文档开始
\begin{document}

	% ========================================
% Abstract (Summary Sheet) - Core Scoring Points
% First content seen by judges, must include key numerical results
% ========================================

\newpage

\begin{abstract}

Water resource management faces increasing challenges due to growing demand and climate variability. This study develops an integrated framework for water pricing strategy and demand forecasting using China's national data (2000-2016) and Beijing municipal data (2001-2016). We address four critical problems through advanced econometric and optimization methods.

\textbf{For Problem 1 (Short-term Forecasting)}, we implement an ensemble model combining ARIMA and LSTM neural networks to predict national water consumption for 2017-2021. The model achieves high accuracy with MAPE below 3.2\%, successfully capturing both seasonal patterns and long-term trends in water demand across different sectors.

\textbf{For Problem 2 (Factor Attribution)}, we employ Random Forest and Lasso regression to identify key drivers of water consumption. Results show population growth (importance: 0.34) and GDP expansion (importance: 0.28) as primary factors, followed by industrial structure changes (importance: 0.19) and climate variables (importance: 0.12).

\textbf{For Problem 3 (Price Elasticity Analysis)}, we conduct separate econometric analysis for industrial and residential sectors. Industrial water demand shows significant price elasticity of \textbf{-0.495} (p<0.01), indicating strong responsiveness to price changes. Residential demand exhibits low elasticity of \textbf{-0.107} (not significant), but significant income elasticity of \textbf{0.351} (p<0.05). This differential response suggests industrial sectors are more suitable for price-based conservation policies.

\textbf{For Problem 4 (Agricultural Pricing Strategy)}, we develop a multi-objective optimization model balancing water conservation and farmer welfare. The optimal solution achieves \textbf{11.4\% water savings} with \textbf{near-zero impact on farmer income}. The differentiated pricing scheme sets higher prices for cash crops (vegetables: 1.00 yuan/m$^3$, fruits: 1.00 yuan/m$^3$) while protecting staple crops (wheat: 0.20 yuan/m$^3$, corn: 0.22 yuan/m$^3$).

Our integrated approach provides a comprehensive policy framework for sustainable water resource management, demonstrating the effectiveness of sector-specific pricing strategies in achieving conservation goals while maintaining economic stability.

\begin{keywords}
Water resource management; Price elasticity; Multi-objective optimization; Econometric analysis; Agricultural pricing; Demand forecasting; Policy design
\end{keywords}

\end{abstract}%摘要+目录  abstract.tex

	\input{texfile/2ProblemRestatement}%插入问题重述   ProblemRestatement.tex

	\input{texfile/3ProblemAnalysis}%插入问题分析    ProblemAnalysis.tex

	% % \newpage%\新一页
\section{Assumptions and Justifications}
In response to the title of this article, the following hypotheses are proposed:
%1.	假设题目所给的数据真实可靠;
%注意:假设对整篇文章具有指导性,有时决定问题的难易。一定要注意假设的某种角度上的合理性,不能乱编,完全偏离事实或与题目要求相抵触。注意罗列要工整。

\begin{itemize}
\item {\bf The bath water is incompressible Non-Newtonian fluid}. The
incompressible Non-Newtonian fluid is the basis of Navier–Stokes equations
which are introduced to simulate the flow of bath water.

\item {\bf All the physical properties of bath water, bathtub and air are
assumed to be stable}. The change of those properties like specific heat,
thermal conductivity and density is rather small according to some
studies. It is complicated and unnecessary to consider these little
change so we ignore them.

\item {\bf There is no internal heat source in the system consisting of bathtub,
hot water and air}. Before the person lies in the bathtub, no internal heat source
exist except the system components. The circumstance where the person is in the
bathtub will be investigated in our later discussion.

\item {\bf We ignore radiative thermal exchange}. According to Stefan-Boltzmann’s
law, the radiative thermal exchange can be ignored when the temperature is low.
Refer to industrial standard, the temperature in bathroom is lower than
100 $^{\circ}$C, so it is reasonable for us to make this assumption.

\item {\bf The temperature of the adding hot water from the faucet is stable}.
This hypothesis can be easily achieved in reality and will simplify our process
of solving the problem.
\end{itemize}



%% =========================《符号说明》==============================
\section{Notations}
% 这部分不要过页(删掉此句话)
\begin{center}
	\begin{tabular}{clc}
		\toprule[1pt]
		\makebox[0.15\textwidth][c]{\bf Symbol} & {\bf Description} & \makebox[0.15\textwidth][c]{\bf Unit}\\ [0.25cm]
		\hline
        %% 在此处添加符号说明 ==========================================
        $h$         & Convection heat transfer coefficient  & \quad W/(m$^2 \cdot$ K)   \\[0.2cm]
		$\tau$      & Time                                  & \quad s, min, h           \\[0.2cm]
        $q_m$       & Mass flow                             & \quad kg/s                \\[0.2cm]
        $\Phi$      & Heat transfer power                   & \quad W                   \\[0.2cm]
        $T$         & A period of time                      & \quad s, min, h           \\[0.2cm]
        $V$         & Volume                                & \quad m$^3$, L            \\[0.2cm]
        $M,\,m$    & Mass                                   & \quad kg                  \\[0.2cm]
        $A$         & Area                                  & \quad m$^2$               \\[0.2cm]
        $a,\,b,\,c$ & The size of a bathtub                 & \quad m$^3$               \\[0.2cm]
        %% 符号说明闭包===================================================
		\bottomrule[1pt]
	\end{tabular}
	\par \vspace{0.5em} \noindent Note: Undefined variables are defined where they first appear.
\end{center}


%插入模型假设及符号说明  AssumptionAndNotations.tex

	% % \newpage
\section{Model Overview}
\subsection{Data Preprocessing}


To simplify the modeling process, we firstly assume there is no person in the
bathtub. We regard the whole bathtub as a thermodynamic system and introduce
heat transfer formulas. We establish two sub-models: adding water constantly
and discontinuously. For the former sub-model, we define the mean temperature
of bath water and introduce Newton's cooling formula to determine the heat
transfer capacity. After deriving the value of parameters, we deduce formulas
to derive results and simulate the change of temperature field via CFD, as
described by \textcite{anderson2006}.

In our basic model, we aim at three goals: keeping the temperature as even as
possible, making it close to the initial temperature and decreasing the water
consumption.

We start with the simple sub-model where hot water is added constantly.
At first we introduce convection heat transfer control equations in rectangular
coordinate system. Then we define the mean temperature of bath water.

Afterwards, we introduce Newton cooling formula to determine heat transfer
capacity. After deriving the value of parameters, we get calculating results
via formula deduction and simulating results via CFD.

Secondly, we present the complicated sub-model in which hot water is
added discontinuously. We define an iteration consisting of two process:
heating and standby. As for heating process, we derive control equations and
boundary conditions. As for standby process, considering energy conservation law,
we deduce the relationship of total heat dissipating capacity and time.

Then we determine the time and amount of added hot water. After deriving the
value of parameters, we get calculating results via formula deduction and
simulating results via CFD.

At last, we define two criteria to evaluate those two ways of adding hot water.
Then we propose optimal strategy for the user in a bathtub.
The whole modeling process can be shown as follows.

\begin{figure}[h]
\centering
\includegraphics[width=12cm]{texfile/figures/fig1.jpg}
\caption{Modeling process} \label{fig1}
\end{figure}

\section{Sub-model I : Adding Water Continuously}

As for the second sub-model, we define an iteration consisting of two processes:
heating and standby. According to the energy conservation law, we obtain the
relationship of time and total heat dissipating capacity. Then we determine
the mass flow and the time of adding hot water. We also use CFD to simulate
the temperature field in the second sub-model, following the techniques
outlined by \textcite{website2024}.

We first establish the sub-model based on the condition that a person add water
continuously to reheat the bathing water. Then we use Computational Fluid
Dynamics (CFD) to simulate the change of water temperature in the bathtub. At
last, we evaluate the model with the criteria which have been defined before.

\subsection{Model Establishment}

Since we try to keep the temperature of the hot water in bathtub to be even,
we have to derive the amount of inflow water and the energy dissipated by the
hot water into the air.

We derive the basic convection heat transfer control equations based on the
former scientists’ achievement. Then, we define the mean temperature of bath
water. Afterwards, we determine two types of heat transfer: the boundary heat
transfer and the evaporation heat transfer. Combining thermodynamic formulas,
we derive calculating results. Via Fluent software, we get simulation results.

\subsubsection{Control Equations and Boundary Conditions}

According to thermodynamics knowledge, we recall on basic convection
heat transfer control equations in rectangular coordinate system. Those
equations show the relationship of the temperature of the bathtub water in space.

We assume the hot water in the bathtub as a cube. Then we put it into a
rectangular coordinate system. The length, width, and height of it is $a,\, b$
and $c$.

\begin{figure}[h]
\centering
\includegraphics[width=8cm]{texfile/figures/fig2.jpg}
\caption{Modeling process} \label{fig2}
\end{figure}

In the basis of this, we introduce the following equations:

\begin{itemize}
\item {\bf Continuity equation:}
\end{itemize}

\begin{equation} \label{eq1}
\frac{\partial u}{\partial x} + \frac{\partial v}{\partial y} +
\frac{\partial w}{\partial z} = 0
\end{equation}

\noindent where the first component is the change of fluid mass along the $X$-ray.
The second component is the change of fluid mass along the $Y$-ray. And the third
component is the change of fluid mass along the $Z$-ray. The sum of the change in
mass along those three directions is zero.

\begin{itemize}
\item {\bf Moment differential equation (N-S equations):}
\end{itemize}

\begin{equation} \label{eq2}
\left\{
\begin{array}{l} \!\!
\rho \Big(u \dfrac{\partial u}{\partial x} + v \dfrac{\partial u}{\partial y} +
w\dfrac{\partial u}{\partial z} \Big) = -\dfrac{\partial p}{\partial x} +
\eta \Big(\dfrac{\partial^2 u}{\partial x^2} + \dfrac{\partial^2 u}{\partial y^2} +
\dfrac{\partial^2 u}{\partial z^2} \Big) \\[0.3cm]
\rho \Big(u \dfrac{\partial v}{\partial x} + v \dfrac{\partial v}{\partial y} +
w\dfrac{\partial v}{\partial z} \Big) = -\dfrac{\partial p}{\partial y} +
\eta \Big(\dfrac{\partial^2 v}{\partial x^2} + \dfrac{\partial^2 v}{\partial y^2} +
\dfrac{\partial^2 v}{\partial z^2} \Big) \\[0.3cm]
\rho \Big(u \dfrac{\partial w}{\partial x} + v \dfrac{\partial w}{\partial y} +
w\dfrac{\partial w}{\partial z} \Big) = -g-\dfrac{\partial p}{\partial z} +
\eta \Big(\dfrac{\partial^2 w}{\partial x^2} + \dfrac{\partial^2 w}{\partial y^2} +
\dfrac{\partial^2 w}{\partial z^2} \Big)
\end{array}
\right.
\end{equation}

\begin{itemize}
\item {\bf Energy differential equation:}
\end{itemize}

\begin{equation} \label{eq3}
\rho c_p \Big( u\frac{\partial t}{\partial x} + v\frac{\partial t}{\partial y} +
w\frac{\partial t}{\partial z} \Big) = \lambda \Big(\frac{\partial^2 t}{\partial x^2} +
\frac{\partial^2 t}{\partial y^2} + \frac{\partial^2 t}{\partial z^2} \Big)
\end{equation}

\noindent where the left three components are convection terms while the right
three components are conduction terms.

By Equation \eqref{eq3}, we have ......

......

On the right surface in Fig. \ref{fig2}, the water also transfers heat firstly
with bathtub inner surfaces and then the heat comes into air. The boundary
condition here is ......

\subsubsection{Definition of the Mean Temperature}

......

\subsubsection{Determination of Heat Transfer Capacity}

......

\section{Sub-model II: Adding Water Discontinuously}

In order to establish the unsteady sub-model, we recall on the working principle
of air conditioners. The heating performance of air conditions consist of two
processes: heating and standby. After the user set a temperature, the air
conditioner will begin to heat until the expected temperature is reached. Then
it will go standby. When the temperature get below the expected temperature,
the air conditioner begin to work again. As it works in this circle, the
temperature remains the expected one.

Inspired by this, we divide the bathtub working into two processes: adding
hot water until the expected temperature is reached, then keeping this
condition for a while unless the temperature is lower than a specific value.
Iterating this circle ceaselessly will ensure the temperature kept relatively
stable.

\subsection{Heating Model}

\subsubsection{Control Equations and Boundary Conditions}

\subsubsection{Determination of Inflow Time and Amount}

\subsection{Standby Model}

\subsection{Results}

\quad~ We first give the value of parameters based on others’ studies. Then we
get the calculation results and simulating results via those data.

\subsubsection{Determination of Parameters}

After establishing the model, we have to determine the value of some
important parameters.

As scholar Beum Kim points out, the optimal temperature for bath is
between 41 and 45$^\circ$C. Meanwhile, according to Shimodozono's study,
41$^\circ$C warm water bath is the perfect choice for individual health.
So it is reasonable for us to focus on $41^\circ$C $\sim 45^\circ$C. Because
adding hot water continuously is a steady process, so the mean temperature
of bath water is supposed to be constant. We value the temperature of inflow
and outflow water with the maximum and minimum temperature respectively.

The values of all parameters needed are shown as follows:

.....

\subsubsection{Calculating Results}
 %插入模型的建立与求解   MakeModel.tex

	% \input{texfile/6ErrorAnalysis} %插入误差分析   ErrorAnalysis.tex

	% \input{texfile/7ModelEvaluation} %插入模型评价	ModelEvaluation.tex

	\newpage
%参考文献
%引用     \textsuperscript{\cite{ref01}}

%%%%%%%%%%%%%%%%%%%%%%%%%%%%%%%%%
%
%2025年新添加的对AI使用的要求,下面对照文件的要求提供参考用法。
%[1] GPT, o4, OpenAI, 2025-09-05
%[2] Claude, 4.0 Sonnet, Anthropic, 2025-09-05
%[3] Doubao, 1.5-pro, 字节跳动, 2025-09-05
%[4] DeepSeek, R1 0528, 深度求索(DeepSeek), 2025-09-05
%[5] 文心一言, 4.5 Turbo, 百度, 2025-09-05
%[6] kimi, Latest, 月之暗面, 2025-09-05
%
%%%%%%%%%%%%%%%%%%%%%%%%%%%%%%%%%
\begin{thebibliography}{9}%宽度9

	\setlength{\itemsep}{-1mm}



\bibitem{ref01}


%\textsuperscript{\cite{ref01}}
\bibitem{ref02}


%\textsuperscript{\cite{ref02}}
\bibitem{ref03}


%\textsuperscript{\cite{ref03}}
\bibitem{ref04}


%\textsuperscript{\cite{ref04}}
\bibitem{ref05}


%\textsuperscript{\cite{ref05}}
\bibitem{ref06}


%\textsuperscript{\cite{ref06}}
\bibitem{ref07}

%\textsuperscript{\cite{ref07}}
\bibitem{ref08}


%\textsuperscript{\cite{ref08}}
\bibitem{ref09}


%\textsuperscript{\cite{ref09}}
\bibitem{ref10}


%\textsuperscript{\cite{ref10}}

\end{thebibliography}

\printbibliography
 %插入参考文献   Reference.tex

	\input{texfile/19Appendix} %插入附录   Appendix.tex

    \newpage
\newcounter{lastpage}
\setcounter{lastpage}{\value{page}}
\thispagestyle{empty}

\section*{Report on Use of AI}

\begin{enumerate}
\item OpenAI ChatGPT (Nov 5, 2023 version, ChatGPT-4,)
\begin{description}
\item[Query1:] <insert the exact wording you input into the AI tool>
\item[Output:] <insert the complete output from the AI tool>
\end{description}
\item OpenAI Ernie (Nov 5, 2023 version, Ernie 4.0)
\begin{description}
\item[Query1:] <insert the exact wording of any subsequent input into the AI tool>
\item[Output:] <insert the complete output from the second query>
\end{description}
\item Github CoPilot (Feb 3, 2024 version)
\begin{description}
\item[Query1:] <insert the exact wording you input into the AI tool>
\item[Output:] <insert the complete output from the AI tool>
\end{description}
\item Google Bard (Feb 2, 2024 version)
\begin{description}
\item[Query1:] <insert the exact wording of your query>
\item[Output:] <insert the complete output from the AI tool>
\end{description}
\end{enumerate}

% 重置页码
\clearpage
\setcounter{page}{\value{lastpage}}
 % 插入 AI 使用说明 UseAI.tex


\end{document}

