\section{Supplementary Tables and Results}

\subsection{Detailed Elasticity Analysis Results}

\begin{table}[h]
\centering
\begin{tabular}{|l|c|c|c|c|c|}
\hline
\textbf{Sector} & \textbf{Price Elasticity} & \textbf{Std. Error} & \textbf{P-value} & \textbf{95\% CI} & \textbf{R²} \\
\hline
Industrial & -0.495 & 0.117 & 0.001*** & [-0.747, -0.243] & 0.993 \\
Residential & -0.107 & 0.132 & 0.430 & [-0.389, 0.175] & 0.983 \\
\hline
\end{tabular}
\caption{Detailed Water Price Elasticity Estimation Results}
\label{tab:detailed_elasticity}
\end{table}

\subsection{Agricultural Optimization: Complete Results}

\begin{table}[h]
\centering
\begin{tabular}{|l|c|c|c|c|c|}
\hline
\textbf{Crop} & \textbf{Optimal Price} & \textbf{Base Water} & \textbf{Optimized Water} & \textbf{Water Savings} & \textbf{Income Impact} \\
\textbf{} & \textbf{(yuan/m³)} & \textbf{(m³/acre)} & \textbf{(m³/acre)} & \textbf{(\%)} & \textbf{(\%)} \\
\hline
Rice & 0.40 & 400 & 373 & 6.8 & -2.1 \\
Wheat & 0.20 & 300 & 325 & -8.4 & +1.8 \\
Corn & 0.22 & 350 & 375 & -7.1 & +1.5 \\
Vegetables & 1.00 & 500 & 328 & 34.4 & -8.2 \\
Fruits & 1.00 & 600 & 371 & 38.2 & -12.1 \\
\hline
\textbf{Overall} & \textbf{-} & \textbf{420} & \textbf{372} & \textbf{11.4} & \textbf{0.0} \\
\hline
\end{tabular}
\caption{Complete Agricultural Water Pricing Optimization Results}
\label{tab:complete_agricultural_results}
\end{table}

\subsection{Model Performance Validation}

\begin{table}[h]
\centering
\begin{tabular}{|c|c|c|c|c|}
\hline
\textbf{Year} & \textbf{Actual (10⁹ m³)} & \textbf{Predicted (10⁹ m³)} & \textbf{Absolute Error} & \textbf{MAPE (\%)} \\
\hline
2017 & 60.434 & 60.76 & 0.33 & 0.55 \\
2018 & 60.155 & 60.80 & 0.64 & 1.07 \\
2019 & 60.212 & 60.83 & 0.62 & 1.02 \\
2020 & 58.129 & 60.84 & 2.71 & 4.67* \\
2021 & 59.202 & 60.85 & 1.65 & 2.78 \\
\hline
\textbf{Average (2017-2019)} & \textbf{-} & \textbf{-} & \textbf{0.53} & \textbf{0.88} \\
\hline
\end{tabular}
\caption{Ensemble Model Validation Results (2017-2021)}
\label{tab:model_validation}
\end{table}

*Note: 2020 deviation attributed to COVID-19 impact

\subsection{Factor Importance Rankings}

\begin{table}[h]
\centering
\begin{tabular}{|l|c|c|c|}
\hline
\textbf{Factor} & \textbf{Importance Score} & \textbf{Rank} & \textbf{Policy Relevance} \\
\hline
Population Growth & 0.34 & 1 & Demographic planning \\
GDP Growth & 0.28 & 2 & Economic development \\
Industrial Structure & 0.19 & 3 & Structural transformation \\
Climate Variables & 0.12 & 4 & Climate adaptation \\
Urbanization Rate & 0.07 & 5 & Urban planning \\
\hline
\end{tabular}
\caption{Water Consumption Driver Importance Analysis}
\label{tab:factor_importance}
\end{table}

\subsection{Regional Price Adjustment Framework}

\begin{table}[h]
\centering
\begin{tabular}{|l|c|c|c|c|}
\hline
\textbf{Region} & \textbf{Water Scarcity} & \textbf{Price Factor} & \textbf{Example Price} & \textbf{Policy Priority} \\
\textbf{} & \textbf{Level} & \textbf{} & \textbf{(Rice, yuan/m³)} & \textbf{} \\
\hline
North China Plain & Severe & 1.2 & 0.48 & High \\
Northwest Arid & Extreme & 1.5 & 0.60 & Highest \\
Yangtze River Basin & Moderate & 0.8 & 0.32 & Medium \\
Northeast Region & Low & 0.7 & 0.28 & Low \\
South China & Abundant & 0.6 & 0.24 & Lowest \\
\hline
\end{tabular}
\caption{Regional Water Pricing Adjustment Framework}
\label{tab:regional_pricing}
\end{table}

\subsection{Sensitivity Analysis Summary}

\begin{table}[h]
\centering
\begin{tabular}{|l|c|c|c|}
\hline
\textbf{Scenario} & \textbf{Parameter Change} & \textbf{Water Savings (\%)} & \textbf{Income Impact (\%)} \\
\hline
Baseline & - & 11.4 & 0.0 \\
Income Shock & -10\% farmer income & 8.2 & -4.8 \\
Mild Drought & +20\% water scarcity & 18.3 & -7.2 \\
Severe Drought & +50\% water scarcity & 25.1 & -12.5 \\
High Elasticity & +20\% price elasticity & 13.8 & -2.1 \\
Low Elasticity & -20\% price elasticity & 9.1 & +1.8 \\
\hline
\end{tabular}
\caption{Sensitivity Analysis Results for Agricultural Pricing}
\label{tab:sensitivity_analysis}
\end{table}
