\subsection{Industrial Water Price Elasticity}

Industrial water price elasticity analysis employs a log-linear demand model considering dual effects of price and GDP:

\begin{equation}
\ln(Q_{industrial}) = \alpha + \beta_1 \ln(P_{water}) + \beta_2 \ln(GDP_{industrial}) + \varepsilon
\end{equation}

where $Q_{industrial}$ represents industrial water consumption, $P_{water}$ is industrial water price, and $GDP_{industrial}$ is industrial GDP.

\subsubsection{Estimation Results}

Using 2001-2016 data, ordinary least squares estimation yields:

\begin{table}[h]
\centering
\begin{tabular}{|l|c|c|c|c|}
\hline
\textbf{Variable} & \textbf{Coefficient} & \textbf{Std. Error} & \textbf{P-value} & \textbf{95\% CI} \\
\hline
Price Elasticity & -0.495 & 0.117 & 0.001*** & [-0.747, -0.243] \\
GDP Elasticity & 0.604 & 0.095 & 0.000*** & [0.398, 0.810] \\
\hline
\multicolumn{5}{|l|}{R-squared: 0.993, Adj. R-squared: 0.992} \\
\hline
\end{tabular}
\caption{Industrial Water Demand Elasticity Estimation Results}
\label{tab:industrial_elasticity}
\end{table}

\subsubsection{Economic Interpretation}

The industrial water price elasticity of -0.495 indicates:
\begin{itemize}
    \item \textbf{Elastic Demand}: Industrial water consumption is sensitive to price changes
    \item \textbf{Efficiency-Forcing Mechanism}: Price increases incentivize firms to invest in water-saving technologies
    \item \textbf{Technical Substitution}: Opportunities exist for water recycling and reuse technologies
    \item \textbf{Cost Sensitivity}: Water costs significantly impact production competitiveness
\end{itemize}

Policy Implications: A 10\% increase in industrial water prices can reduce industrial water consumption by approximately 5.0\%, demonstrating significant effectiveness of price policies for industrial water conservation.