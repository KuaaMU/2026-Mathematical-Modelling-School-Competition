\subsection{Factor Ranking and Importance Analysis}

\subsubsection{Grey Relational Analysis Results}

Since the dataset contains only 17 samples, we utilize Grey Relational Analysis (GRA) to evaluate the correlation strength of macro-drivers with total water consumption. The GRA method is particularly suitable for small sample sizes and captures non-linear relationships.

\begin{table}[h]
\centering
\begin{tabular}{|l|c|c|}
\hline
\textbf{Factor} & \textbf{GRA Score} & \textbf{Ranking} \\
\hline
Population & 0.965 & 1 \\
GDP & 0.556 & 2 \\
Industrial Structure & 0.487 & 3 \\
Urbanization Rate & 0.423 & 4 \\
Agricultural Output & 0.398 & 5 \\
Climate Variables & 0.312 & 6 \\
\hline
\end{tabular}
\caption{Grey Relational Analysis Factor Ranking}
\label{tab:gra_ranking}
\end{table}

\subsubsection{Random Forest Feature Importance}

To complement the GRA analysis, we employ Random Forest regression to assess feature importance:

\begin{itemize}
    \item \textbf{Population Growth}: 0.34 (highest importance)
    \item \textbf{GDP Expansion}: 0.28 (second highest)
    \item \textbf{Industrial Structure}: 0.19 (moderate importance)
    \item \textbf{Climate Variables}: 0.12 (lower importance)
    \item \textbf{Other Factors}: 0.07 (minimal importance)
\end{itemize}

\subsubsection{Economic Interpretation}

The results indicate that \textbf{population growth} has significantly higher correlation with water usage (GRA=0.965) than GDP (GRA=0.556). This suggests several important insights:

\begin{enumerate}
    \item \textbf{Demographic Driver Dominance}: Population expansion directly drives residential water demand and indirectly affects agricultural and industrial needs
    \item \textbf{Decoupling Evidence}: The relatively lower GDP correlation suggests successful decoupling of economic growth from water consumption intensity
    \item \textbf{Efficiency Gains}: Industrial upgrading and water-saving technologies have reduced the water-GDP elasticity over time
    \item \textbf{Policy Effectiveness}: Water conservation policies have successfully broken the traditional water-economic growth linkage
\end{enumerate}

\subsubsection{Temporal Stability Analysis}

Rolling window analysis reveals changing factor importance over time:

\begin{itemize}
    \item \textbf{2000-2007}: GDP dominance period (correlation > 0.8)
    \item \textbf{2008-2012}: Transition period with mixed drivers
    \item \textbf{2013-2016}: Population dominance period (correlation > 0.9)
\end{itemize}

This temporal shift reflects China's economic transformation from extensive to intensive growth patterns.