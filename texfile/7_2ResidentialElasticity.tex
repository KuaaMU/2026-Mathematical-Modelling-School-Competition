\subsection{Residential Water Price Elasticity}

Residential water price elasticity analysis employs a demand model incorporating income effects:

\begin{equation}
\ln(Q_{residential}) = \alpha + \beta_1 \ln(P_{water}) + \beta_2 \ln(Income) + \varepsilon
\end{equation}

where $Q_{residential}$ represents residential water consumption, $P_{water}$ is residential water price, and $Income$ is per capita disposable income.

\subsubsection{Estimation Results}

\begin{table}[h]
\centering
\begin{tabular}{|l|c|c|c|c|}
\hline
\textbf{Variable} & \textbf{Coefficient} & \textbf{Std. Error} & \textbf{P-value} & \textbf{95\% CI} \\
\hline
Price Elasticity & -0.107 & 0.132 & 0.430 & [-0.389, 0.175] \\
Income Elasticity & 0.351 & 0.154 & 0.031** & [0.301, 0.393] \\
\hline
\multicolumn{5}{|l|}{R-squared: 0.983, Adj. R-squared: 0.980} \\
\hline
\end{tabular}
\caption{Residential Water Demand Elasticity Estimation Results}
\label{tab:residential_elasticity}
\end{table}

\subsubsection{Economic Interpretation}

Residential water analysis reveals:

\textbf{Price Effects}:
\begin{itemize}
    \item Price elasticity of -0.107 indicates inelastic demand
    \item Statistically insignificant (p=0.430), suggesting limited price impact on residential consumption
    \item Reflects essential nature of residential water use and habit dependency
\end{itemize}

\textbf{Income Effects}:
\begin{itemize}
    \item Income elasticity of 0.351 is statistically significant (p=0.031)
    \item Indicates water is a normal good with consumption increasing with income
    \item Income effects dominate price effects, consistent with developing country characteristics
\end{itemize}

\subsubsection{Affordability Analysis}

Residential water affordability analysis shows:
\begin{itemize}
    \item Water expenses represent small share of household disposable income (typically <2\%)
    \item Low price sensitivity primarily due to small expenditure share
    \item Tiered pricing needed to protect low-income households
\end{itemize}