\subsection{Parameter Stability Analysis}

\subsubsection{Elasticity Parameter Robustness}

We conduct Monte Carlo simulation to test the stability of our optimization results under parameter uncertainty:

\textbf{Simulation Setup}:
\begin{itemize}
    \item 1000 random draws from normal distributions around estimated elasticity values
    \item Standard deviation set at 20\% of point estimates
    \item Re-optimize pricing for each parameter set
\end{itemize}

\textbf{Results}:
\begin{table}[h]
\centering
\begin{tabular}{|l|c|c|c|}
\hline
\textbf{Parameter} & \textbf{Mean} & \textbf{Std Dev} & \textbf{95\% CI} \\
\hline
Rice Price (yuan/m$^3$) & 0.39 & 0.04 & [0.32, 0.47] \\
Wheat Price (yuan/m$^3$) & 0.21 & 0.03 & [0.16, 0.26] \\
Water Savings (\%) & 11.1 & 1.8 & [7.8, 14.2] \\
Income Impact (\%) & 0.2 & 0.4 & [-0.5, 1.1] \\
\hline
\end{tabular}
\caption{Parameter Stability Test Results}
\label{tab:parameter_stability}
\end{table}

\subsubsection{Model Structure Sensitivity}

We test alternative model specifications:

\begin{enumerate}
    \item \textbf{Linear vs. Log-Linear Demand}: Results differ by less than 8\%
    \item \textbf{Alternative Constraint Levels}: 6\% vs. 10\% affordability constraint changes optimal prices by 12-15\%
    \item \textbf{Different Crop Aggregations}: 3-crop vs. 7-crop models yield similar policy recommendations
\end{enumerate}

\textbf{Conclusion}: The optimization framework demonstrates robust performance across reasonable parameter ranges and model specifications.