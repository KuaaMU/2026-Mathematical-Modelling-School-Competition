% ========================================
% 文件:5_3ResultsAndVerification.tex
% 预测结果与验证 (Results and Verification)
% 重点:利用2017-2021真值进行“反事实分析”
% ========================================

\subsection{Results Analysis and Model Verification}

To rigorously evaluate the performance of our optimized ensemble model (ARIMA + Polynomial Regression),
we conducted a \textbf{hindcasting validation} using the ground truth data from 2017 to 2021. This period
is critical as it covers both "normal years" and the "extreme shock year" (COVID-19), allowing us to
test both the accuracy and the sensitivity of the model.

\subsubsection{Model Robustness in Normal Years (2017-2019)}

As shown in Table \ref{tab:validation_results}, the model demonstrates exceptional accuracy during the
pre-pandemic period. The relative errors for 2017, 2018, and 2019 are \textbf{0.5\%}, \textbf{1.1\%},
and \textbf{1.0\%}, respectively.

This high fidelity indicates that our model successfully captures the "Saturation Phase" characteristics
of China's water consumption. Unlike exponential growth models (e.g., GM(1,1)) which tend to overestimate,
our ensemble approach correctly identifies the stabilizing trend driven by water-saving policies and
industrial upgrading.

\subsubsection{Quantification of COVID-19 Impact (Counterfactual Analysis)}

A significant divergence is observed in 2020, as visualized in Figure \ref{fig:forecast_validation}.
\begin{itemize}
    \item \textbf{The Counterfactual Baseline:} The model predicted a water consumption of \textbf{60.84 billion $m^3$} for 2020. This value represents the \textit{counterfactual scenario}—i.e., the expected consumption level if the pandemic had not occurred.
    \item \textbf{The Real-world Shock:} The actual consumption dropped to \textbf{58.13 billion $m^3$}.
    \item \textbf{Impact Quantification:} The \textbf{4.7\% gap} between the prediction and the actual value quantitatively measures the negative impact of COVID-19 on industrial production and social activities.
\end{itemize}

In 2021, the error narrowed to \textbf{2.8\%}, suggesting a "V-shaped" or partial recovery of economic
activities, though a lagged effect remains.

\begin{figure}[htbp]
    \centering
    % 请确保文件名和你保存的图片一致
    \includegraphics[width=1.0\textwidth]{code/Q1/question1/results/forecast_final_no_gm.png}
    \caption{\textbf{Model Validation and Impact Quantification (2017-2021).} The red dashed line represents the model's baseline prediction (Counterfactual Scenario). The low error rates ($<$1.1\%) during 2017-2019 validate the model's robustness. The significant gap in 2020 (4.7\%) quantitatively reflects the external shock caused by COVID-19.}
    \label{fig:forecast_validation}
\end{figure}

\subsubsection{Detailed Validation Data}

The detailed comparison between the predicted values (Ensemble Model) and the ground truth is presented
in Table \ref{tab:validation_results}.

\begin{table}[h]
\centering
\caption{Comparison of Predicted vs. Actual Water Consumption (2017-2021)}
\label{tab:validation_results}
\begin{tabular}{ccccc}
\toprule
\textbf{Year} & \textbf{Actual ($10^9 m^3$)} & \textbf{Predicted ($10^9 m^3$)} & \textbf{Abs. Error ($10^9 m^3$)} & \textbf{Rel. Error (\%)} \\
\midrule
2017 & 60.434 & 60.76 & 0.33 & \textbf{0.55\%} \\
2018 & 60.155 & 60.80 & 0.64 & \textbf{1.07\%} \\
2019 & 60.212 & 60.83 & 0.62 & \textbf{1.02\%} \\
2020 & \textit{58.129} & \textit{60.84} & \textit{2.71} & \textit{\textbf{4.67\% (Shock)}} \\
2021 & 59.202 & 60.85 & 1.65 & 2.78\% \\
\bottomrule
\end{tabular}
\end{table}

\textbf{Conclusion:} The Mean Absolute Percentage Error (MAPE) for normal years (2017-2019)
is \textbf{0.88\%}, far below the standard threshold of 3\%. This validates that the model is
highly reliable for future forecasting under stable socioeconomic conditions.
