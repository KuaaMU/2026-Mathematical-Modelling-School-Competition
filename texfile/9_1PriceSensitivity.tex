% ========================================
% 价格敏感性分析 (Price Sensitivity)
% 重点测试农业定价的稳健性
% ========================================

\subsection{Price Sensitivity Analysis}
\subsubsection{Farmer Income Shock Test}
假设农民收入下降10\%(如农产品价格下跌),重新优化定价策略:

% \begin{figure}[h]
% \centering
% \includegraphics[width=0.7\textwidth]{income_shock.pdf}
% \caption{农民收入下降10\%时的最优水价调整}
% \label{fig:income_shock}
% \end{figure}

\textbf{结果}:基础水价需从0.35元/m³降至0.30元/m³,第一阶梯阈值从450m³/亩提高到500m³/亩,以维持农民收入下降不超过5\%。

\subsubsection{Water Scarcity Stress Test}
在极端干旱条件下(可用水量减少20\%),水价调整方案:

% \begin{table}[h]
% \centering
% \begin{tabular}{|c|c|c|c|}
% \hline
% \textbf{情景} & \textbf{基础水价(元/m³)} & \textbf{节水效果} & \textbf{农民收入影响} \\
% \hline
% 基准情景 & 0.35 & 12.7\% & -4.8\% \\
% \hline
% 轻度干旱 & 0.45 & 18.3\% & -7.2\% \\
% \hline
% 严重干旱 & 0.60 & 25.1\% & -12.5\% \\
% \hline
% \end{tabular}
% \caption{不同干旱情景下的水价调整}
% \label{tab:drought_scenarios}
% \end{table}

\textbf{结论}:水价策略具有较强适应性,可根据水资源状况动态调整,但需配套建立\textbf{干旱应急补贴机制}。

% =============== 写作指导 ===============
% 1. 必须测试极端情景(收入下降、干旱)
% 2. 用图表展示敏感性结果(如上图和表)
% 3. 给出具体的调整方案,不是泛泛而谈
% 4. 强调配套机制的重要性
% 5. 敏感性分析要量化(具体数字)
