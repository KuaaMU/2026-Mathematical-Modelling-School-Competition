% ========================================
% Consolidated Sensitivity and Robustness Analysis
% ========================================

\subsection{Sensitivity Analysis and Model Robustness}

\subsubsection{Agricultural Pricing Robustness}

We test pricing strategy robustness under three scenarios:

\textbf{Income Shock Test}: 10\% farmer income decline requires price adjustment (base price: 0.35→0.30 yuan/m³) while maintaining 8.2\% water savings and <5\% income impact.

\textbf{Water Scarcity Stress}: Under severe drought (20\% water reduction), pricing adjusts to 0.60 yuan/m³ achieving 25.1\% water savings with 12.5\% income impact, requiring emergency subsidies.

\textbf{Parameter Sensitivity}: ±20\% elasticity variation changes optimal prices by <15\%, demonstrating robust performance across parameter ranges.

\subsubsection{Model Parameter Stability}

Elasticity coefficient stability tests show:
\begin{itemize}
    \item Industrial elasticity: Stable within [-0.4, -0.6] range across different time periods
    \item Residential elasticity: Consistently insignificant, confirming inelastic nature
    \item Agricultural elasticity: Crop-specific variations within expected bounds
\end{itemize}

\subsubsection{System Robustness Under Extreme Conditions}

Climate stress testing demonstrates adaptive capacity through dynamic pricing adjustments, but requires complementary policy measures (subsidies, technology support) to maintain farmer welfare during extreme events.