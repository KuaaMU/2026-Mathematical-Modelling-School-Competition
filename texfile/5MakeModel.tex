% \newpage
\section{Model Overview}
\subsection{Data Preprocessing}


To simplify the modeling process, we firstly assume there is no person in the
bathtub. We regard the whole bathtub as a thermodynamic system and introduce
heat transfer formulas. We establish two sub-models: adding water constantly
and discontinuously. For the former sub-model, we define the mean temperature
of bath water and introduce Newton's cooling formula to determine the heat
transfer capacity. After deriving the value of parameters, we deduce formulas
to derive results and simulate the change of temperature field via CFD, as
described by \textcite{anderson2006}.

In our basic model, we aim at three goals: keeping the temperature as even as
possible, making it close to the initial temperature and decreasing the water
consumption.

We start with the simple sub-model where hot water is added constantly.
At first we introduce convection heat transfer control equations in rectangular
coordinate system. Then we define the mean temperature of bath water.

Afterwards, we introduce Newton cooling formula to determine heat transfer
capacity. After deriving the value of parameters, we get calculating results
via formula deduction and simulating results via CFD.

Secondly, we present the complicated sub-model in which hot water is
added discontinuously. We define an iteration consisting of two process:
heating and standby. As for heating process, we derive control equations and
boundary conditions. As for standby process, considering energy conservation law,
we deduce the relationship of total heat dissipating capacity and time.

Then we determine the time and amount of added hot water. After deriving the
value of parameters, we get calculating results via formula deduction and
simulating results via CFD.

At last, we define two criteria to evaluate those two ways of adding hot water.
Then we propose optimal strategy for the user in a bathtub.
The whole modeling process can be shown as follows.

\begin{figure}[h]
\centering
\includegraphics[width=12cm]{texfile/figures/fig1.jpg}
\caption{Modeling process} \label{fig1}
\end{figure}

\section{Sub-model I : Adding Water Continuously}

As for the second sub-model, we define an iteration consisting of two processes:
heating and standby. According to the energy conservation law, we obtain the
relationship of time and total heat dissipating capacity. Then we determine
the mass flow and the time of adding hot water. We also use CFD to simulate
the temperature field in the second sub-model, following the techniques
outlined by \textcite{website2024}.

We first establish the sub-model based on the condition that a person add water
continuously to reheat the bathing water. Then we use Computational Fluid
Dynamics (CFD) to simulate the change of water temperature in the bathtub. At
last, we evaluate the model with the criteria which have been defined before.

\subsection{Model Establishment}

Since we try to keep the temperature of the hot water in bathtub to be even,
we have to derive the amount of inflow water and the energy dissipated by the
hot water into the air.

We derive the basic convection heat transfer control equations based on the
former scientists’ achievement. Then, we define the mean temperature of bath
water. Afterwards, we determine two types of heat transfer: the boundary heat
transfer and the evaporation heat transfer. Combining thermodynamic formulas,
we derive calculating results. Via Fluent software, we get simulation results.

\subsubsection{Control Equations and Boundary Conditions}

According to thermodynamics knowledge, we recall on basic convection
heat transfer control equations in rectangular coordinate system. Those
equations show the relationship of the temperature of the bathtub water in space.

We assume the hot water in the bathtub as a cube. Then we put it into a
rectangular coordinate system. The length, width, and height of it is $a,\, b$
and $c$.

\begin{figure}[h]
\centering
\includegraphics[width=8cm]{texfile/figures/fig2.jpg}
\caption{Modeling process} \label{fig2}
\end{figure}

In the basis of this, we introduce the following equations:

\begin{itemize}
\item {\bf Continuity equation:}
\end{itemize}

\begin{equation} \label{eq1}
\frac{\partial u}{\partial x} + \frac{\partial v}{\partial y} +
\frac{\partial w}{\partial z} = 0
\end{equation}

\noindent where the first component is the change of fluid mass along the $X$-ray.
The second component is the change of fluid mass along the $Y$-ray. And the third
component is the change of fluid mass along the $Z$-ray. The sum of the change in
mass along those three directions is zero.

\begin{itemize}
\item {\bf Moment differential equation (N-S equations):}
\end{itemize}

\begin{equation} \label{eq2}
\left\{
\begin{array}{l} \!\!
\rho \Big(u \dfrac{\partial u}{\partial x} + v \dfrac{\partial u}{\partial y} +
w\dfrac{\partial u}{\partial z} \Big) = -\dfrac{\partial p}{\partial x} +
\eta \Big(\dfrac{\partial^2 u}{\partial x^2} + \dfrac{\partial^2 u}{\partial y^2} +
\dfrac{\partial^2 u}{\partial z^2} \Big) \\[0.3cm]
\rho \Big(u \dfrac{\partial v}{\partial x} + v \dfrac{\partial v}{\partial y} +
w\dfrac{\partial v}{\partial z} \Big) = -\dfrac{\partial p}{\partial y} +
\eta \Big(\dfrac{\partial^2 v}{\partial x^2} + \dfrac{\partial^2 v}{\partial y^2} +
\dfrac{\partial^2 v}{\partial z^2} \Big) \\[0.3cm]
\rho \Big(u \dfrac{\partial w}{\partial x} + v \dfrac{\partial w}{\partial y} +
w\dfrac{\partial w}{\partial z} \Big) = -g-\dfrac{\partial p}{\partial z} +
\eta \Big(\dfrac{\partial^2 w}{\partial x^2} + \dfrac{\partial^2 w}{\partial y^2} +
\dfrac{\partial^2 w}{\partial z^2} \Big)
\end{array}
\right.
\end{equation}

\begin{itemize}
\item {\bf Energy differential equation:}
\end{itemize}

\begin{equation} \label{eq3}
\rho c_p \Big( u\frac{\partial t}{\partial x} + v\frac{\partial t}{\partial y} +
w\frac{\partial t}{\partial z} \Big) = \lambda \Big(\frac{\partial^2 t}{\partial x^2} +
\frac{\partial^2 t}{\partial y^2} + \frac{\partial^2 t}{\partial z^2} \Big)
\end{equation}

\noindent where the left three components are convection terms while the right
three components are conduction terms.

By Equation \eqref{eq3}, we have ......

......

On the right surface in Fig. \ref{fig2}, the water also transfers heat firstly
with bathtub inner surfaces and then the heat comes into air. The boundary
condition here is ......

\subsubsection{Definition of the Mean Temperature}

......

\subsubsection{Determination of Heat Transfer Capacity}

......

\section{Sub-model II: Adding Water Discontinuously}

In order to establish the unsteady sub-model, we recall on the working principle
of air conditioners. The heating performance of air conditions consist of two
processes: heating and standby. After the user set a temperature, the air
conditioner will begin to heat until the expected temperature is reached. Then
it will go standby. When the temperature get below the expected temperature,
the air conditioner begin to work again. As it works in this circle, the
temperature remains the expected one.

Inspired by this, we divide the bathtub working into two processes: adding
hot water until the expected temperature is reached, then keeping this
condition for a while unless the temperature is lower than a specific value.
Iterating this circle ceaselessly will ensure the temperature kept relatively
stable.

\subsection{Heating Model}

\subsubsection{Control Equations and Boundary Conditions}

\subsubsection{Determination of Inflow Time and Amount}

\subsection{Standby Model}

\subsection{Results}

\quad~ We first give the value of parameters based on others’ studies. Then we
get the calculation results and simulating results via those data.

\subsubsection{Determination of Parameters}

After establishing the model, we have to determine the value of some
important parameters.

As scholar Beum Kim points out, the optimal temperature for bath is
between 41 and 45$^\circ$C. Meanwhile, according to Shimodozono's study,
41$^\circ$C warm water bath is the perfect choice for individual health.
So it is reasonable for us to focus on $41^\circ$C $\sim 45^\circ$C. Because
adding hot water continuously is a steady process, so the mean temperature
of bath water is supposed to be constant. We value the temperature of inflow
and outflow water with the maximum and minimum temperature respectively.

The values of all parameters needed are shown as follows:

.....

\subsubsection{Calculating Results}
