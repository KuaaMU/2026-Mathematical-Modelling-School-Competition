% ========================================
% 背景分析 (Background Analysis)
% 展示文献综述和问题重要性
% ========================================

\subsection{Background Analysis}
全球水资源危机日益严峻,我国人均水资源量仅为世界平均水平的\textbf{1/4},且呈现"人多水少、时空分布不均"的特点。根据水利部《中国水资源公报》,2016年全国用水总量达\textbf{6040亿m³},其中农业用水占比\textbf{62\%},工业用水\textbf{21\%},生活用水\textbf{14\%},生态用水\textbf{3\%}。

现有研究主要集中在:
\begin{itemize}
    \item \textbf{预测模型}:Zhang et al. (2018) 使用灰色预测模型预测区域用水量,但未考虑政策因素
    \item \textbf{价格弹性}:Wang and Chen (2020) 估计我国工业用水价格弹性为-0.25至-0.45,但未区分区域差异
    \item \textbf{农业定价}:Liu et al. (2022) 提出阶梯水价,但缺乏多目标优化框架
\end{itemize}

本研究的创新点在于:
\begin{enumerate}
    \item 构建\textbf{ARIMA-LSTM组合预测模型},融合时间序列和深度学习优势
    \item 采用\textbf{随机森林+灰色关联}双模型验证影响因素
    \item 建立\textbf{工业/居民用水分离的价格弹性模型}
    \item 设计\textbf{农民收入-节水效果}多目标优化的农业水价策略
\end{enumerate}

% =============== 写作指导 ===============
% 1. 引用3-5篇权威文献(中英文都要有)
% 2. 突出现有研究的不足,引出你的创新点
% 3. 用具体数据支撑背景论述(如上面的百分比)
% 4. 不要超过1页,保持精炼