% ========================================
% 文件:2_2BackgroundAnalysis.tex
% 背景分析 (Background Analysis) - 针对 Q1/Q2 实际方案优化版
% ========================================

\subsection{Background Analysis}
Global water scarcity has become an increasingly critical constraint on sustainable development. China, in particular, faces a paradoxical challenge: while it possesses significant total water resources, its per capita availability is only about \textbf{1/4} of the global average. According to the \textit{China Water Resources Bulletin}, the total national water consumption in 2016 reached \textbf{604.0 billion $m^3$}, with a highly skewed distribution: agriculture (\textbf{62\%}), industry (\textbf{21\%}), domestic use (\textbf{14\%}), and ecological maintenance (\textbf{3\%}).

Recent scholarly efforts have focused on several key dimensions:
\begin{itemize}
    \item \textbf{Forecasting Paradigms:} Traditional studies (Zhang et al., 2018) often utilize GM(1,1) or single linear models. However, these fail to account for the "Saturation Phase" observed in China's water usage since 2013, where consumption has shifted from rapid growth to a stable plateau.
    \item \textbf{Driving Factors:} Existing research (Wang \& Chen, 2020) primarily uses correlation analysis, which often falls into the "identity trap" by regressing total usage against its own components (e.g., industrial usage), thus obscuring the deeper socio-economic drivers like GDP and technology.
    \item \textbf{Agricultural Pricing:} While Liu et al. (2022) proposed tiered pricing, few have quantified the delicate trade-off between water conservation and farmers' disposable income within a multi-objective optimization framework.
\end{itemize}

\subsection{Research Innovations of This Study}
In light of the gaps identified above, this study proposes an integrated framework with the following innovations:
\begin{enumerate}
    \item \textbf{Robust Ensemble Forecasting with Hindcasting Validation:}
    Instead of complex deep learning which risks overfitting on small samples, we build an \textbf{ARIMA-Polynomial Ensemble Model}. We innovatively use a \textbf{"Counterfactual Analysis"} approach to validate the model's baseline against 2017-2021 data, effectively quantifying the external shock of events like COVID-19.

    \item \textbf{Dual-Track Attribution and Decoupling Analysis:}
    We move beyond simple correlation by combining \textbf{Grey Relational Analysis (GRA)} with \textbf{Standardized Regression}. This allows us to identify the \textbf{"Relative Decoupling"} effect between GDP growth and water consumption, revealing how efficiency gains counteract scale expansion.

    \item \textbf{Differentiated Price Response Modeling:}
    We establish separate price elasticity models for industrial and residential sectors to capture their unique sensitivities to water costs and income levels.

    \item \textbf{Pareto-Optimal Agricultural Strategy:}
    We design an agricultural pricing policy using a \textbf{Multi-Objective Optimization} model, ensuring a scientifically-grounded balance between environmental sustainability and social equity.
\end{enumerate}
