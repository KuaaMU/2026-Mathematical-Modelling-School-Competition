% ========================================
% Abstract (Summary Sheet) - Core Scoring Points
% First content seen by judges, must include key numerical results
% ========================================

\newpage

\begin{abstract}

Water resource management faces increasing challenges due to growing demand and climate variability. This study develops an integrated framework for water pricing strategy and demand forecasting using China's national data (2000-2016) and Beijing municipal data (2001-2016). We address four critical problems through advanced econometric and optimization methods.

\textbf{For Problem 1 (Short-term Forecasting)}, we implement an ensemble model combining ARIMA and LSTM neural networks to predict national water consumption for 2017-2021. The model achieves high accuracy with MAPE below 3.2\%, successfully capturing both seasonal patterns and long-term trends in water demand across different sectors.

\textbf{For Problem 2 (Factor Attribution)}, we employ Random Forest and Lasso regression to identify key drivers of water consumption. Results show population growth (importance: 0.34) and GDP expansion (importance: 0.28) as primary factors, followed by industrial structure changes (importance: 0.19) and climate variables (importance: 0.12).

\textbf{For Problem 3 (Price Elasticity Analysis)}, we conduct separate econometric analysis for industrial and residential sectors. Industrial water demand shows significant price elasticity of \textbf{-0.495} (p<0.01), indicating strong responsiveness to price changes. Residential demand exhibits low elasticity of \textbf{-0.107} (not significant), but significant income elasticity of \textbf{0.351} (p<0.05). This differential response suggests industrial sectors are more suitable for price-based conservation policies.

\textbf{For Problem 4 (Agricultural Pricing Strategy)}, we develop a multi-objective optimization model balancing water conservation and farmer welfare. The optimal solution achieves \textbf{11.4\% water savings} with \textbf{near-zero impact on farmer income}. The differentiated pricing scheme sets higher prices for cash crops (vegetables: 1.00 yuan/m$^3$, fruits: 1.00 yuan/m$^3$) while protecting staple crops (wheat: 0.20 yuan/m$^3$, corn: 0.22 yuan/m$^3$).

Our integrated approach provides a comprehensive policy framework for sustainable water resource management, demonstrating the effectiveness of sector-specific pricing strategies in achieving conservation goals while maintaining economic stability.

\begin{keywords}
Water resource management; Price elasticity; Multi-objective optimization; Econometric analysis; Agricultural pricing; Demand forecasting; Policy design
\end{keywords}

\end{abstract}