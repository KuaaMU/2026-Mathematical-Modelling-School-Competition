% ========================================
% 数据来源 (Data Sources)
% 必须详细说明每个数据的来源
% ========================================

\subsection{Data Sources}
本研究使用的数据来源如下:

\begin{table}[h]
\centering
\begin{tabular}{|l|l|l|}
\hline
\textbf{数据类型} & \textbf{来源} & \textbf{时间范围} \\
\hline
全国用水量 & 《中国统计年鉴2017》 & 2000-2016 \\
\hline
人口、GDP & 国家统计局官网 & 2000-2016 \\
\hline
工业/生活水价 & 《中国水资源公报》 & 2005-2016 \\
\hline
北京市数据 & 《北京市统计年鉴》 & 2001-2016 \\
\hline
农业用水成本 & 农业农村部调研报告 & 2015 \\
\hline
\end{tabular}
\caption{数据来源说明表}
\label{tab:data_sources}
\end{table}

\textbf{补充数据说明}:由于附件数据中2000-2004年水价缺失,我们从《中国价格年鉴》补全;农业用水成本数据通过农业农村部《全国农产品成本收益资料汇编》获取。

% =============== 写作指导 ===============
% 1. 必须包含数据来源表格(如上)
% 2. 每个数据都要有具体来源(网站URL或书名)
% 3. 说明缺失数据的处理方法
% 4. 农业用水成本是关键,必须找到可靠来源