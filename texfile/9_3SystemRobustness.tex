\subsection{System Robustness Analysis}

\subsubsection{Climate Change Adaptation}

We evaluate the pricing system's performance under climate change scenarios:

\textbf{Temperature Increase Scenarios}:
\begin{itemize}
    \item \textbf{+1°C}: 5\% increase in crop water requirements
    \item \textbf{+2°C}: 12\% increase in crop water requirements  
    \item \textbf{+3°C}: 20\% increase in crop water requirements
\end{itemize}

\textbf{Adaptation Strategy}:
\begin{table}[h]
\centering
\begin{tabular}{|c|c|c|c|}
\hline
\textbf{Temperature Rise} & \textbf{Price Adjustment} & \textbf{Water Savings} & \textbf{Adaptation Cost} \\
\hline
+1°C & +15\% & 8.5\% & Low \\
+2°C & +35\% & 6.2\% & Moderate \\
+3°C & +60\% & 3.8\% & High \\
\hline
\end{tabular}
\caption{Climate Adaptation Pricing Adjustments}
\label{tab:climate_adaptation}
\end{table}

\subsubsection{Economic Shock Resilience}

Testing system performance under macroeconomic shocks:

\begin{itemize}
    \item \textbf{GDP Decline (-10\%)}: Pricing system maintains feasibility with reduced conservation targets
    \item \textbf{Inflation (+20\%)}: Automatic adjustment mechanisms preserve real affordability
    \item \textbf{Agricultural Crisis}: Emergency protocols activate alternative pricing tiers
\end{itemize}

\subsubsection{Technology Integration Capacity}

Evaluating system adaptability to technological advances:

\begin{itemize}
    \item \textbf{Smart Irrigation}: 30\% efficiency gains allow 25\% price reduction while maintaining conservation
    \item \textbf{Drought-Resistant Crops}: Reduced water requirements enable more aggressive pricing
    \item \textbf{Precision Agriculture}: Real-time optimization potential for dynamic pricing
\end{itemize}

\textbf{Conclusion}: The pricing framework demonstrates strong systemic robustness with built-in adaptation mechanisms for climate, economic, and technological changes.