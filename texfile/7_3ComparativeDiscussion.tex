\subsection{Comparative Analysis: Industrial vs Residential}

\subsubsection{Elasticity Comparison}

Comparative analysis of industrial and residential water price elasticity:

\begin{table}[h]
\centering
\begin{tabular}{|l|c|c|c|}
\hline
\textbf{Sector} & \textbf{Price Elasticity} & \textbf{Statistical Significance} & \textbf{Demand Type} \\
\hline
Industrial & -0.495 & Significant (p<0.01) & Elastic \\
Residential & -0.107 & Not Significant (p=0.43) & Inelastic \\
\hline
\end{tabular}
\caption{Price Elasticity Comparison Between Sectors}
\label{tab:elasticity_comparison}
\end{table}

\subsubsection{Economic Mechanisms}

\textbf{Mechanisms Behind High Industrial Elasticity}:
\begin{enumerate}
    \item \textbf{Efficiency-Forcing Mechanism}: Price increases incentivize firms to invest in water-saving technologies
    \item \textbf{Technical Substitution Possibilities}: Opportunities exist for water recycling and reuse systems
    \item \textbf{Production Cost Sensitivity}: Water costs affect firm competitiveness, promoting conservation
    \item \textbf{Competitive Pressure}: Market competition requires firms to control all production costs
\end{enumerate}

\textbf{Mechanisms Behind Low Residential Elasticity}:
\begin{enumerate}
    \item \textbf{Essential Need Rigidity}: Basic needs like drinking and washing are difficult to compress
    \item \textbf{Habit Dependency}: Water consumption behaviors exhibit strong habitual characteristics
    \item \textbf{Limited Substitutes}: Residential water use lacks effective substitutes
    \item \textbf{Small Income Share}: Water expenses represent small proportion of household income
\end{enumerate}

\subsubsection{Policy Implications}

Policy recommendations based on elasticity analysis:

\textbf{Industrial Water Policy}:
\begin{itemize}
    \item Prioritize industrial water price regulation with significant effects
    \item 10\% industrial water price increase can reduce consumption by approximately 5.0\%
    \item Promote water-saving technologies and recycling systems
    \item Establish water efficiency standards and incentive mechanisms
\end{itemize}

\textbf{Residential Water Policy}:
\begin{itemize}
    \item Price policy effects are limited, requiring complementary measures
    \item Implement tiered pricing to protect basic water needs
    \item Strengthen water conservation education to change consumption habits
    \item Provide subsidies for water-saving appliances and technical support
\end{itemize}

\subsubsection{Conservation Potential Ranking}

Based on price elasticity analysis, water conservation potential ranks as:
\begin{enumerate}
    \item \textbf{Industrial Sector}: High potential with significant price policy effectiveness
    \item \textbf{Residential Sector}: Moderate potential requiring comprehensive policy measures
\end{enumerate}

These results provide scientific basis for developing differentiated water pricing policies.