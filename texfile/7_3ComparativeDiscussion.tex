\subsection{Policy Implications and Sector Prioritization}

Comparative analysis reveals fundamental differences in price responsiveness:

\begin{table}[h]
\centering
\begin{tabular}{|l|c|c|c|}
\hline
\textbf{Sector} & \textbf{Price Elasticity} & \textbf{Significance} & \textbf{Policy Priority} \\
\hline
Industrial & -0.495 & p<0.01 & High \\
Residential & -0.107 & p=0.43 & Medium \\
\hline
\end{tabular}
\caption{Sector-Specific Price Elasticity and Policy Priorities}
\label{tab:elasticity_comparison}
\end{table}

\subsubsection{Differentiated Policy Framework}

\textbf{Industrial Focus}: Price-based policies are highly effective (10\% price increase → 5.0\% consumption reduction). Prioritize industrial water pricing with technology incentives.

\textbf{Residential Approach}: Price policies have limited effect. Implement tiered pricing protecting basic needs, combined with education and appliance subsidies.

This analysis provides scientific basis for sector-specific water conservation strategies, maximizing policy effectiveness through targeted interventions.