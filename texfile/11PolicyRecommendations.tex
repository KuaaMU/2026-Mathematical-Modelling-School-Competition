% ========================================
% 政策建议 (Policy Recommendations)
% 必须具体、可操作、有时间表
% ========================================

\subsection{Policy Recommendations}
\subsubsection{Short-term Measures (2025-2026)}
\begin{itemize}
    \item \textbf{工业用水}:在长三角、珠三角试点阶梯水价,2025年Q1启动,基础水价提高15\%(从3.5元/m³到4.0元/m³)
    \item \textbf{农业用水}:在华北平原试点本方案阶梯水价,同步发放每亩50元节水补贴
    \item \textbf{监测体系}:建立国家级水资源监测平台,实时跟踪用水量变化
\end{itemize}

\subsubsection{Medium-term Planning (2027-2029)}
\begin{itemize}
    \item \textbf{水权交易}:建立跨省水权交易市场,允许节余用水指标交易,价格区间2.0-5.0元/m³
    \item \textbf{技术推广}:中央财政投入200亿元,推广高效灌溉技术,目标2029年覆盖80\%耕地
    \item \textbf{法规完善}:修订《水法》,明确农业用水收费法律依据
\end{itemize}

\subsubsection{Long-term Mechanism (2030+)}
\begin{itemize}
    \item \textbf{水资源GDP}:将水资源消耗纳入地方政府考核,建立"水资源GDP"核算体系
    \item \textbf{生态补偿}:建立流域生态补偿机制,上游节水地区获得下游补偿
    \item \textbf{气候适应}:将气候变化影响纳入水资源规划,建立弹性水价调整机制
\end{itemize}

\textbf{Implementation Roadmap}:
% \begin{figure}[h]
% \centering
% \includegraphics[width=0.9\textwidth]{policy_timeline.pdf}
% \caption{政策实施路线图}
% \label{fig:policy_timeline}
% \end{figure}

% =============== 写作指导 ===============
% 1. 按短/中/长期分阶段
% 2. 每个建议都要有具体数字(金额、百分比、时间)
% 3. 必须包含实施路线图(甘特图)
% 4. 优先级:先试点再推广
% 5. 考虑财政可行性(不要建议不切实际的投入)
