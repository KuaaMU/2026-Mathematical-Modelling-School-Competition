% 模型假设内容
% \newpage%\新一页
\section{Assumptions and Justifications}
In response to the title of this article, the following hypotheses are proposed:
%1.	假设题目所给的数据真实可靠;
%注意:假设对整篇文章具有指导性,有时决定问题的难易。一定要注意假设的某种角度上的合理性,不能乱编,完全偏离事实或与题目要求相抵触。注意罗列要工整。

\begin{itemize}
\item {\bf The bath water is incompressible Non-Newtonian fluid}. The
incompressible Non-Newtonian fluid is the basis of Navier–Stokes equations
which are introduced to simulate the flow of bath water.

\item {\bf All the physical properties of bath water, bathtub and air are
assumed to be stable}. The change of those properties like specific heat,
thermal conductivity and density is rather small according to some
studies. It is complicated and unnecessary to consider these little
change so we ignore them.

\item {\bf There is no internal heat source in the system consisting of bathtub,
hot water and air}. Before the person lies in the bathtub, no internal heat source
exist except the system components. The circumstance where the person is in the
bathtub will be investigated in our later discussion.

\item {\bf We ignore radiative thermal exchange}. According to Stefan-Boltzmann’s
law, the radiative thermal exchange can be ignored when the temperature is low.
Refer to industrial standard, the temperature in bathroom is lower than
100 $^{\circ}$C, so it is reasonable for us to make this assumption.

\item {\bf The temperature of the adding hot water from the faucet is stable}.
This hypothesis can be easily achieved in reality and will simplify our process
of solving the problem.
\end{itemize}
