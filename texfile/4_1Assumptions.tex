% 模型假设内容
% \newpage%\新一页
% \section{Assumptions and Justifications}
To simplify the complex real-world problem into a mathematically solvable form, we make the following reasonable assumptions based on the problem context:

\begin{itemize}
    \item \textbf{Assumption 1: Data Credibility and Continuity.}
    We assume that the data provided in the attachment and the supplementary data collected from the National Bureau of Statistics are authentic and reliable. Although there are minor statistical discrepancies in certain years, they do not affect the overall macroscopic trend. Outliers caused by data recording errors are handled during preprocessing.

    \item \textbf{Assumption 2: Inertia of Socio-Economic Development.}
    We assume that the social and economic development of the country follows a relatively continuous trend. While short-term shocks (like COVID-19 in 2020) exist, the underlying mechanisms driving water consumption (e.g., population growth, industrialization) do not undergo catastrophic structural changes overnight. This validates the use of time-series models (ARIMA) and regression analysis.

    \item \textbf{Assumption 3: Dominance of Selected Factors.}
    In the factor analysis (Problem 2), we assume that Population, GDP, and the internal structure of water usage (Agricultural/Industrial ratios) are the primary drivers of total water consumption. Other minor factors, such as slight annual variations in rainfall or specific local policies, are considered negligible or captured within the random error term ($\epsilon$) of the model.

    \item \textbf{Assumption 4: Rational Economic Behavior.}
    In the pricing strategy analysis (Problem 3 \& 4), we assume that water users (both industrial factories and residents) represent "rational economic agents." This means their water consumption behavior is sensitive to price changes, following the Law of Demand: as water price increases, consumption decreases (negative price elasticity), provided other conditions remain constant.

    \item \textbf{Assumption 5: Independence of Sub-sectors.}
    We assume that the pricing mechanisms for industrial, residential, and agricultural water are relatively independent in policy implementation, although they share the total available water resources. This allows us to model their price elasticities separately.
\end{itemize}
