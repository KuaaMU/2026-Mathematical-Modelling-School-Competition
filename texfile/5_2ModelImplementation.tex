% ========================================
% 文件:5_2ModelImplementation.tex
% 模型实现细节 (Model Implementation)
% 重点:展示数学公式,定义ARIMA和回归模型,以及权重的分配
% ========================================

\subsection{Model Implementation}

Based on the selection strategy, we implemented the AutoRegressive Integrated Moving Average (ARIMA) model and the Polynomial Regression model, integrating them into a final predictive framework.

\subsubsection{ARIMA Model Construction}
The ARIMA($p, d, q$) model combines autoregression (AR), differencing (I), and moving average (MA).

Let $Y_t$ denote the total water consumption at year $t$. To ensure stationarity, we apply differencing of order $d$:
\begin{equation}
    Y'_t = (1-B)^d Y_t
\end{equation}
where $B$ is the backshift operator, defined as $B Y_t = Y_{t-1}$.

The general form of the ARIMA model is expressed as:
\begin{equation}
    \left(1 - \sum_{i=1}^p \phi_i B^i\right) Y'_t = c + \left(1 + \sum_{j=1}^q \theta_j B^j\right) \epsilon_t
\end{equation}
where:
\begin{itemize}
    \item $\phi_i$ are the autoregressive parameters (AR part).
    \item $\theta_j$ are the moving average parameters (MA part).
    \item $\epsilon_t \sim N(0, \sigma^2)$ is the white noise error term.
    \item $c$ is a constant.
\end{itemize}

Using the \textbf{Grid Search} method based on the \textbf{AIC (Akaike Information Criterion)}, we determined the optimal hyperparameters to be ARIMA(1, 1, 0), which effectively captures the short-term fluctuations.

\subsubsection{Polynomial Regression Construction}
To capture the macroscopic trend of "growth to saturation," we employ a quadratic polynomial regression. The model hypothesis is:
\begin{equation}
    Y_t = \beta_0 + \beta_1 t + \beta_2 t^2 + \xi_t
\end{equation}
where $t$ represents the time index (Year), $\beta_0, \beta_1, \beta_2$ are regression coefficients estimated using the Ordinary Least Squares (OLS) method, and $\xi_t$ is the random error.
The quadratic term $\beta_2 t^2$ is crucial as it allows the model to simulate the deceleration of water consumption growth observed in recent years.

\subsubsection{Ensemble Model Integration}
The final prediction $\hat{Y}_{final}$ is obtained by a weighted linear combination of the two base models:
\begin{equation}
    \hat{Y}_{final}(t) = w_1 \cdot \hat{Y}_{ARIMA}(t) + w_2 \cdot \hat{Y}_{Poly}(t)
\end{equation}
subject to the constraint $w_1 + w_2 = 1$.

\textbf{Weight Optimization:}
Instead of assigning equal weights, we optimized the weights based on the validation performance on the most recent training data (2013-2016). We observed that:
\begin{itemize}
    \item The ARIMA model provides high stability for short-term steps.
    \item The Polynomial model captures the overall curvature but carries extrapolation risks.
\end{itemize}
Consequently, we assigned a higher weight to the statistical time-series model. The final weights were determined as:
\begin{equation}
    w_{ARIMA} = 0.70, \quad w_{Poly} = 0.30
\end{equation}
This configuration prioritizes stationarity while retaining the non-linear trend component. Note that the GM(1,1) model was assigned a weight of 0 (effectively removed) as it failed to capture the saturation characteristic.
