\subsection{Implementation Strategy and Expected Outcomes}

\subsubsection{Optimal Price Structure}

The multi-objective optimization yields differentiated pricing protecting staple crops while targeting high water-consuming cash crops:

\begin{table}[h]
\centering
\begin{tabular}{|l|c|c|c|}
\hline
\textbf{Crop Type} & \textbf{Optimal Price (yuan/m³)} & \textbf{Water Savings (\%)} & \textbf{Policy Rationale} \\
\hline
Rice & 0.40 & 6.8 & Moderate conservation \\
Wheat & 0.20 & -8.4* & Food security protection \\
Corn & 0.22 & -7.1* & Food security protection \\
Vegetables & 1.00 & 34.4 & High conservation target \\
Fruits & 1.00 & 38.2 & High conservation target \\
\hline
\end{tabular}
\caption{Optimal Agricultural Water Pricing Scheme}
\label{tab:optimal_pricing}
\end{table}

*Increased water allocation for staple crops ensures food security while overall system achieves 11.4\% water savings.

\begin{figure}[h]
\centering
\includegraphics[width=0.8\textwidth]{code/Q4/results/test_visualization_improvements.png}
\caption{Agricultural Water Pricing Implementation Strategy}
\label{fig:implementation_strategy}
\end{figure}

\subsubsection{Phased Implementation}

Three-phase rollout (2025-2030) with gradual price increases (70\%→85\%→100\% of optimal) and supporting measures including farmer subsidies, technology promotion, and monitoring systems. Expected outcomes: 11.4\% water conservation with minimal farmer income impact.