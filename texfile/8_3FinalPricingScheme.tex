\subsection{Final Pricing Scheme}

\subsubsection{Optimal Price Structure}

Based on multi-objective optimization results, we determine optimal water prices for each crop:

\begin{table}[h]
\centering
\begin{tabular}{|l|c|c|c|c|}
\hline
\textbf{Crop Type} & \textbf{Optimal Price} & \textbf{Base Water Use} & \textbf{Optimized Use} & \textbf{Water Savings} \\
\textbf{} & \textbf{(yuan/m$^3$)} & \textbf{(m$^3$/acre)} & \textbf{(m$^3$/acre)} & \textbf{(\%)} \\
\hline
Rice & 0.40 & 400 & 373 & 6.8 \\
Wheat & 0.20 & 300 & 325 & -8.4* \\
Corn & 0.22 & 350 & 375 & -7.1* \\
Vegetables & 1.00 & 500 & 328 & 34.4 \\
Fruits & 1.00 & 600 & 371 & 38.2 \\
\hline
\end{tabular}
\caption{Optimal Agricultural Water Pricing Scheme}
\label{tab:optimal_pricing}
\end{table}

*Note: Increased water use for staple crops due to lower prices, but still satisfies food security constraints.

\subsubsection{Tiered Pricing Structure}

Implementation of tiered pricing system, using rice as example:

\begin{table}[h]
\centering
\begin{tabular}{|c|c|c|c|}
\hline
\textbf{Tier} & \textbf{Usage Range (m$^3$/acre)} & \textbf{Price (yuan/m$^3$)} & \textbf{Application} \\
\hline
Basic Tier & 0-300 & 0.32 & Guarantee basic water needs \\
Standard Tier & 301-450 & 0.40 & Normal production water \\
Conservation Tier & >450 & 0.52 & Encourage water conservation \\
\hline
\end{tabular}
\caption{Tiered Pricing Structure (Rice Example)}
\label{tab:tiered_pricing}
\end{table}

\subsubsection{Regional Adjustment Mechanism}

Price adjustment factors based on regional water resource endowments:

\begin{table}[h]
\centering
\begin{tabular}{|l|c|c|}
\hline
\textbf{Region} & \textbf{Water Resource Status} & \textbf{Price Factor} \\
\hline
North China Plain & Severely water-scarce & 1.2 \\
Yangtze River Basin & Relatively abundant & 0.8 \\
Northwest Arid Region & Extremely water-scarce & 1.5 \\
Northeast Region & Relatively abundant & 0.7 \\
South China Region & Abundant & 0.6 \\
\hline
\end{tabular}
\caption{Regional Price Adjustment Factors}
\label{tab:regional_adjustment}
\end{table}

\subsubsection{Implementation Strategy}

\textbf{Phased Implementation Plan}:
\begin{enumerate}
    \item \textbf{Pilot Phase (2025-2026)}: Select 5 provinces for pilot, prices at 70\% of optimal
    \item \textbf{Expansion Phase (2027-2028)}: Extend nationwide, prices at 85\% of optimal
    \item \textbf{Full Implementation (2029-2030)}: Complete implementation at optimal price levels
\end{enumerate}

\textbf{Supporting Measures}:
\begin{itemize}
    \item \textbf{Subsidy Mechanism}: Provide 50 yuan/acre subsidy for farmers with annual income below 10,000 yuan
    \item \textbf{Technical Support}: Promote drip irrigation and sprinkler systems with 50\% government equipment subsidies
    \item \textbf{Monitoring System}: Establish agricultural water metering and monitoring infrastructure
    \item \textbf{Adjustment Mechanism}: Review and adjust price levels every 3 years based on implementation results
\end{itemize}

\subsubsection{Expected Outcomes}

Anticipated implementation effects:
\begin{itemize}
    \item \textbf{Water Conservation}: 11.4\% reduction in total agricultural water use
    \item \textbf{Economic Impact}: Minimal impact on farmer income (<0.1\%)
    \item \textbf{Structural Optimization}: Significant reduction in high water-consuming cash crop irrigation
    \item \textbf{Technology Adoption}: Promote large-scale adoption of water-saving technologies
\end{itemize}