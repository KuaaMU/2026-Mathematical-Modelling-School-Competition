% ========================================
% 最终定价方案 (Final Pricing Scheme)
% 必须包含阶梯水价表和帕累托前沿图
% ========================================

\subsection{Final Pricing Scheme}
\subsubsection{Tiered Pricing Table}
基于多目标优化结果,提出农业阶梯水价方案:

% \begin{table}[h]
% \centering
% \begin{tabular}{|c|c|c|c|c|}
% \hline
% \textbf{用水类型} & \textbf{阶梯} & \textbf{用量阈值(m³/亩)} & \textbf{水价(元/m³)} & \textbf{适用区域} \\
% \hline
% \multirow{3}{*}{粮食作物} & 基础 & 0-300 & 0.35 & 全国 \\
% \cline{2-5}
% & 第一阶梯 & 301-450 & 0.65 & 缺水地区 \\
% \cline{2-5}
% & 第二阶梯 & >450 & 0.85 & 严重缺水地区 \\
% \hline
% \multirow{2}{*}{经济作物} & 基础 & 0-400 & 0.45 & 全国 \\
% \cline{2-5}
% & 超量 & >400 & 1.10 & 所有地区 \\
% \hline
% \end{tabular}
% \caption{农业阶梯水价方案}
% \label{tab:agricultural_pricing}
% \end{table}

\subsubsection{Pareto Frontier Analysis}
多目标优化的帕累托前沿如图\ref{fig:pareto_frontier}所示:

% \begin{figure}[h]
% \centering
% \includegraphics[width=0.8\textwidth]{pareto_frontier.pdf}
% \caption{农民收入下降 vs 节水效果的帕累托前沿}
% \label{fig:pareto_frontier}
% \end{figure}

\textbf{方案选择}:选择图中红点对应方案(节水12.7\%,农民收入下降4.8\%),该方案位于帕累托前沿的"拐点",在节水效果和农民负担间取得最佳平衡。

\subsubsection{Implementation Strategy}
\begin{itemize}
    \item \textbf{缓冲期}:2025-2026年为试点期,基础水价维持0.30元/m³
    \item \textbf{补贴机制}:对低收入农户(年人均收入<1万元)提供每亩50元补贴
    \item \textbf{技术配套}:同步推广滴灌技术,政府补贴设备费用的50\%
\end{itemize}

% =============== 写作指导 ===============
% 1. 必须包含阶梯水价表格(区分作物类型)
% 2. 必须展示帕累托前沿图(X轴:收入下降,Y轴:节水)
% 3. 必须用红点标注最终选择的方案
% 4. 需要说明实施策略(缓冲期、补贴、技术配套)
% 5. 水价不能过高(一般不超过0.9元/m³)
