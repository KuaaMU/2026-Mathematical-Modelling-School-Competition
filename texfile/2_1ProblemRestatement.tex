% ========================================
% 问题重述 (Problem Restatement)
% 用流程图展示问题分解思路
% ========================================

\subsection{Problem Restatement}
本题要求基于我国2000-2016年全国数据和2001-2016年北京市数据,解决四个核心问题:
\begin{enumerate}
    \item \textbf{短期预测}:预测2017-2021年全国用水量
    \item \textbf{归因分析}:识别人口、GDP、农业/工业/生活/生态用水等因素中的主要影响因素
    \item \textbf{机制分析}:分别研究水价变化对工业用水和居民生活用水量的影响
    \item \textbf{策略设计}:为农业用水设计合理定价策略,平衡节水效果与农民承受能力
\end{enumerate}

% \begin{figure}[h]
% \centering
% \includegraphics[width=0.9\textwidth]{problem_framework.pdf}
% \caption{问题分解框架:从数据到政策}
% \label{fig:problem_framework}
% \end{figure}

% =============== 写作指导 ===============
% 1. 必须包含流程图(如上图),用draw.io或Visio制作
% 2. 流程图要体现:数据输入 → 模型处理 → 政策输出
% 3. 问题描述要简洁,重点突出四个问题的逻辑关系
% 4. 不要复制题目原文,要用自己的话重新组织
