% ========================================
% Problem Restatement
% Use flowchart to show problem decomposition approach
% ========================================

\subsection{Problem Restatement}

This study addresses four interconnected challenges in water resource management using China's national data (2000-2016) and Beijing municipal data (2001-2016):

\begin{enumerate}
    \item \textbf{Short-term Forecasting}: Predict national water consumption for 2017-2021 across all sectors
    \item \textbf{Factor Attribution}: Identify key drivers among population, GDP, agricultural/industrial/residential/ecological water use factors
    \item \textbf{Mechanism Analysis}: Investigate differential impacts of water price changes on industrial versus residential water consumption
    \item \textbf{Policy Design}: Develop optimal agricultural water pricing strategies that balance conservation objectives with farmer affordability
\end{enumerate}

The problems form a logical progression from descriptive analysis (forecasting and attribution) to explanatory analysis (elasticity mechanisms) and finally to prescriptive analysis (optimal pricing policy). This comprehensive approach enables evidence-based policy recommendations for sustainable water resource management.

\subsubsection{Problem Interconnections}

The four problems are strategically connected:
\begin{itemize}
    \item \textbf{Problems 1 \& 2} establish baseline understanding of water demand patterns and driving factors
    \item \textbf{Problem 3} reveals behavioral mechanisms underlying demand responses to price signals
    \item \textbf{Problem 4} applies these insights to design optimal pricing policies for the agricultural sector
\end{itemize}

This integrated framework ensures that policy recommendations are grounded in empirical evidence about demand patterns, factor influences, and price responsiveness across different user sectors.