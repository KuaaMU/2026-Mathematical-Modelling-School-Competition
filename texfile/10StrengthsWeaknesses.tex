\subsection{Model Strengths}

\subsubsection{Methodological Innovations}
\begin{itemize}
    \item \textbf{Integrated Framework}: Successfully combines forecasting, attribution analysis, elasticity modeling, and optimization in a coherent analytical framework
    \item \textbf{Sector-Specific Analysis}: Differentiates between industrial and residential water demand mechanisms, revealing distinct price responsiveness patterns
    \item \textbf{Multi-Objective Optimization}: Balances competing objectives (water conservation vs. farmer welfare) using Pareto frontier analysis
    \item \textbf{Empirical Validation}: Uses robust econometric methods with statistical significance testing and confidence intervals
\end{itemize}

\subsubsection{Policy Relevance}
\begin{itemize}
    \item \textbf{Actionable Results}: Provides specific pricing recommendations with quantified impacts (11.4\% water savings, minimal income effects)
    \item \textbf{Implementation Pathway}: Offers phased implementation strategy with supporting measures and monitoring mechanisms
    \item \textbf{Regional Adaptability}: Includes regional adjustment factors reflecting local water resource conditions
    \item \textbf{Constraint Satisfaction}: Ensures food security and affordability constraints are met in optimization
\end{itemize}

\subsection{Model Limitations}

\subsubsection{Data and Scope Constraints}
\begin{itemize}
    \item \textbf{Limited Time Series}: Analysis based on 16-year dataset may not capture long-term structural changes
    \item \textbf{Aggregation Level}: National and municipal-level analysis may mask important sub-regional variations
    \item \textbf{Crop Simplification}: Agricultural model uses 5 representative crops, potentially overlooking regional crop diversity
    \item \textbf{Static Parameters}: Assumes constant elasticity coefficients, which may vary over time and across regions
\end{itemize}

\subsubsection{Methodological Assumptions}
\begin{itemize}
    \item \textbf{Rational Behavior}: Assumes perfect rational response to price signals, which may not hold in practice
    \item \textbf{Ceteris Paribus}: Elasticity analysis assumes other factors remain constant, limiting real-world applicability
    \item \textbf{Linear Relationships}: Some models assume linear relationships that may be non-linear in reality
    \item \textbf{Technology Neutrality}: Does not explicitly model technological innovation effects on water efficiency
\end{itemize}

\subsection{Model Extensions and Future Research}

\subsubsection{Potential Improvements}
\begin{itemize}
    \item \textbf{Dynamic Modeling}: Incorporate time-varying parameters and adaptive learning mechanisms
    \item \textbf{Spatial Heterogeneity}: Develop region-specific models accounting for local conditions
    \item \textbf{Technology Integration}: Explicitly model water-saving technology adoption and diffusion
    \item \textbf{Behavioral Economics}: Incorporate behavioral factors and bounded rationality in decision-making
\end{itemize}

\subsubsection{Broader Applications}
\begin{itemize}
    \item \textbf{International Adaptation}: Framework can be adapted to other countries with similar water scarcity challenges
    \item \textbf{Climate Integration}: Can be extended to incorporate climate change scenarios and adaptation strategies
    \item \textbf{Multi-Resource Analysis}: Methodology applicable to other natural resource management problems
    \item \textbf{Real-Time Implementation}: Can be integrated with IoT and big data for dynamic pricing systems
\end{itemize}