\subsection{Economic Interpretation of Factor Analysis}

\subsubsection{The Decoupling Phenomenon}

Further analysis using Standardized Linear Regression reveals a profound economic insight regarding the 'Decoupling Effect':

\begin{itemize}
    \item \textbf{Population as a Rigid Driver (Coefficient = 2.306)}: The coefficient for population is dominant and positive. This confirms that basic living needs and food security (driven by population) create a rigid demand for water resources.
    \item \textbf{Relative Decoupling of GDP (Coefficient = 0.137)}: Although China's GDP has grown exponentially, its impact coefficient on water usage is remarkably low (0.137) compared to population. This phenomenon indicates \textbf{Relative Decoupling}. It implies that economic growth is no longer heavily reliant on extensive water consumption. The widespread adoption of water-saving technologies in industry and the shift towards the service sector have successfully improved the \textbf{marginal water use efficiency} of the economy.
\end{itemize}

\subsubsection{Structural Decomposition of Water Consumption}

To understand the composition of water usage, we analyzed the temporal evolution of four major sectors. Agricultural water consistently dominates the consumption structure, accounting for over 60\% of the total volume. Meanwhile, industrial water shows a trend of stabilization after 2010, reflecting the initial success of industrial water-saving policies.

\subsubsection{Identification of Macro-Drivers: Population vs. GDP}

Although structural analysis reveals \textit{where} the water goes, it does not explain \textit{what drives} the total demand. We employed Grey Relational Analysis (GRA) and Standardized Regression to quantify the impact of external macro-factors.

The standardized regression results present a compelling economic insight:

\begin{itemize}
    \item \textbf{Population as a Rigid Driver (Coefficient ≈ 2.31)}: The impact of population is overwhelmingly positive and significant. A 1-unit increase in standardized population leads to a 2.31-unit surge in water consumption. This confirms that demographic expansion creates a rigid demand for basic living and food security (agricultural water).

    \item \textbf{Relative Decoupling of GDP (Coefficient ≈ 0.14)}: In sharp contrast, the coefficient for GDP is remarkably low (0.137). Despite China's exponential economic growth during this period, its marginal impact on water consumption is minimal. This phenomenon is known as \textbf{"Relative Decoupling"}. It indicates that economic growth is shifting from water-intensive industries to high-tech and service sectors, significantly improving the \textbf{marginal water productivity}.
\end{itemize}

\subsubsection{Policy Implications}

The decoupling analysis suggests several important policy directions:

\begin{enumerate}
    \item \textbf{Demographic Planning}: Population growth remains the primary driver, requiring integrated water-population planning
    \item \textbf{Efficiency Focus}: Continue promoting water-saving technologies to maintain the decoupling trend
    \item \textbf{Structural Optimization}: Accelerate transition to water-efficient economic sectors
    \item \textbf{Regional Coordination}: Balance population distribution with water resource availability
\end{enumerate}